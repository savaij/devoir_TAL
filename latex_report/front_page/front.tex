
%%%%%%%%%%%%%%%%%%%%%% DOCUMENT
		\begin{center}
			
			\bigskip
			
			\begin{large}				
				ÉCOLE NATIONALE DES CHARTES\\
				UNIVERSITÉ PARIS, SCIENCES \& LETTRES
			\end{large}
			\begin{center}\rule{2cm}{0.02cm}\end{center}
			
			\bigskip
			\bigskip
			\bigskip
			\begin{large}
				Francesco Paolo Savatteri\\
				\vspace{1.3\baselineskip}
				January 2024
			\end{large}
			%selon le cas
			
			\vspace{4\baselineskip}
			
			\begin{Large}
				Brilliant and mysterious: a brief account of lacking transparency in artificial intelligence
			\end{Large}
			
			\vspace{8\baselineskip}
			
			\begin{normalsize}
			\begin{abstract}
				We are in the middle of the “AI boom”. New models of artificial intelligence are released every day at an unprecedented speed. The implications are profound and very difficult to predict. At the moment, however, it is becoming increasingly necessary to address the lack of transparency in the field: first from a theoretical point of view, in order to be able to define the aspects to be considered; then from a practical point of view, through the emblematic case study of Stability AI.
			\end{abstract}
			
			\end{normalsize}
			\bigskip
			\bigskip
	
	
			\vfill
			
			
		\end{center}

	
	\thispagestyle{empty}	
	\cleardoublepage