L'approche présentée dans ces pages est incomplète pour plusieurs raisons. \\ 
Tout d'abord, la méthode du groupe de contrôle décrite dans la section précédente n'est utile qu'à un niveau général et intuitif. Pour comparer les résultats de manière plus rigoureuse, il existe des méthodes mathématiques qui permettent de mesurer le niveau de \emph{clustering} des données. En outre, la performance d'un modèle sentence-BERT peut elle-même être calculée en utilisant des jeux de données déjà existants, même en italien, afin d'obtenir des scores précis. L'approche décrite dans ces pages n'est donc utile que pour une analyse préliminaire des données dans les cas où un niveau de précision particulièrement élevé n'est pas requis.\\

En outre, un autre aspect n'est pas pris en compte : la présence de valeurs aberrantes. Comme on peut le voir dans la figure 2, certains points sont très éloignés des autres (il y en a aussi dans le cas de la PCA, bien qu'ils ne soient pas visibles dans le graphique). Cela suggère qu'il pourrait s'agir de valeurs aberrantes. Une approche plus complète devrait identifier et décider comment traiter ces valeurs. Une méthode possible, par exemple, consiste à calculer un embedding moyen et à déterminer la distance de chaque texte par rapport à la moyenne, afin d'éliminer ensuite les valeurs trop élevées – la définition de “trop élevé” pouvant être trouvée de différentes manières. 