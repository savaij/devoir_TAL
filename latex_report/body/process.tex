Le prétraitement de ces données est divisé en deux étapes principales. La première concerne des aspects plus simples : l'élimination des données nulles et la vérification de la longueur des textes en excluant les valeurs aberrantes. La deuxième concerne le sens de ces textes, afin de vérifier l'homogénéité de leur contenu. Aux fins du présent article, il convient de se concentrer uniquement sur la deuxième phase.
\\

L'objectif final est de vérifier que les textes du dataset, bien que provenant de sources différentes, ont des significations similaires. Pour ce faire, nous avons d'abord transformé les phrases en un vecteur numérique - ce que l'on appelle "embedding" dans le domaine du traitement automatique des langues. Cette transformation a été effectuée à l'aide d'un modèle de sentence embeddings dérivé du modèle de langue BERT. L'architecture de sentence-BERT et ses détails techniques sont présentés dans {\color{red} \href{https://arxiv.org/abs/1908.10084}{cet article}}.     